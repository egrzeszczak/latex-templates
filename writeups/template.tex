%%  
%%  A write-up template made for LaTeX
%%  
%%  Author: @egrzeszczak
%%  

\documentclass[12pt]{article}

%%  ========================================================================
%%  Set paper geometry
%%  ========================================================================

\usepackage[a4paper, inner=1.5cm, outer=1.5cm, top=2cm, bottom=2cm]{geometry}

%%  ========================================================================
%%  Set font
%%  ========================================================================

\usepackage{fontspec}
\setmainfont{Arimo}

%%  ========================================================================
%%  Set automatic indent to 0px
%%  ========================================================================

\setlength{\parindent}{0pt} 

%%  ========================================================================
%%  Use pictures
%%  ========================================================================

\usepackage{graphicx}

%%  ========================================================================
%%  Page layout
%%  ========================================================================

\usepackage{fancyhdr}
\usepackage{lastpage}
\pagestyle{fancy}
\fancyhf{}

\renewcommand{\headrulewidth}{0pt}
\renewcommand{\footrulewidth}{0pt}

%%  ========================================================================
%%  TLP Settings (https://www.first.org/tlp/docs/tlp-a4.pdf)
%%  ========================================================================

\usepackage{xcolor}
\definecolor{tlpclear}{RGB}{255, 255, 255}
\definecolor{tlpgreen}{RGB}{51, 255, 0}
\definecolor{tlpamber}{RGB}{255, 192, 0}
\definecolor{tlpred}{RGB}{255, 43, 43}

\fancyfoot[C]{Page \thepage \hspace{1pt} of \pageref{LastPage}}

\newcommand{\TLP}{TLP:CLEAR}                                                                      % uncomment for TLP:CLEAR
\fancyhead[R]{\footnotesize \colorbox{black}{\textcolor{tlpclear}{\textbf{\TLP}}}}                % uncomment for TLP:CLEAR
\fancyfoot[R]{\footnotesize \colorbox{black}{\textcolor{tlpclear}{\textbf{\TLP}}}}                % uncomment for TLP:CLEAR

% \newcommand{\TLP}{TLP:GREEN}                                                                    % uncomment for TLP:GREEN
% \fancyhead[R]{\footnotesize \colorbox{black}{\textcolor{tlpgreen}{\textbf{\TLP}}}}              % uncomment for TLP:GREEN
% \fancyfoot[R]{\footnotesize \colorbox{black}{\textcolor{tlpgreen}{\textbf{\TLP}}}}              % uncomment for TLP:GREEN

% \newcommand{\TLP}{TLP:AMBER}                                                                    % uncomment for TLP:AMBER
% \fancyhead[R]{\footnotesize \colorbox{black}{\textcolor{tlpamber}{\textbf{\TLP}}}}              % uncomment for TLP:AMBER
% \fancyfoot[R]{\footnotesize \colorbox{black}{\textcolor{tlpamber}{\textbf{\TLP}}}}              % uncomment for TLP:AMBER

% \newcommand{\TLP}{TLP:AMBER+STRICT}                                                             % uncomment for TLP:AMBER+STRICT
% \fancyhead[R]{\footnotesize \colorbox{black}{\textcolor{tlpamber}{\textbf{\TLP}}}}              % uncomment for TLP:AMBER+STRICT
% \fancyfoot[R]{\footnotesize \colorbox{black}{\textcolor{tlpamber}{\textbf{\TLP}}}}              % uncomment for TLP:AMBER+STRICT

% \newcommand{\TLP}{TLP:RED}                                                                      % uncomment for TLP:RED
% \fancyhead[R]{\footnotesize \colorbox{black}{\textcolor{tlpred}{\textbf{\TLP}}}}                % uncomment for TLP:RED
% \fancyfoot[R]{\footnotesize \colorbox{black}{\textcolor{tlpred}{\textbf{\TLP}}}}                % uncomment for TLP:RED

%%  ========================================================================
%%  Timeline
%%  ========================================================================

\usepackage{environ}
\usepackage{tikz}
\usetikzlibrary{calc,matrix}

%% Code by Claudio:
%% https://tex.stackexchange.com/a/197447/221452
%% Uses code by Andrew:
%% http://tex.stackexchange.com/a/28452/13304
\makeatletter
    \let\matamp=&
    \catcode`\&=13
    \def&{%
        \iftikz@is@matrix%
            \pgfmatrixnextcell%
        \else%
            \matamp%
        \fi%
    }
\makeatother

\newcounter{lines}
\def\endlr{\stepcounter{lines}\\}

\newcounter{vtml}
\setcounter{vtml}{0}

\newif\ifvtimelinetitle
\newif\ifvtimebottomline

\tikzset{
    description/.style={column 2/.append style={#1}},
    timeline color/.store in=\vtmlcolor,
    timeline color=red!80!black,
    timeline color st/.style={fill=\vtmlcolor,draw=\vtmlcolor},
    use timeline header/.is if=vtimelinetitle,
    use timeline header=false,
    add bottom line/.is if=vtimebottomline,
    add bottom line=false,
    timeline title/.store in=\vtimelinetitle,
    timeline title={},
    line offset/.store in=\lineoffset,
    line offset=4pt,
}

\NewEnviron{vtimeline}[1][]{%
    \setcounter{lines}{1}%
    \stepcounter{vtml}%
    \begin{tikzpicture}[column 1/.style={anchor=east},
        column 2/.style={anchor=west},
        text depth=0pt,text height=1ex,
        row sep=1ex,
        column sep=1em,
        #1
    ]
        \matrix(vtimeline\thevtml)[matrix of nodes]{\BODY};
        \pgfmathtruncatemacro\endmtx{\thelines-1}

        \path[timeline color st]
            ($(vtimeline\thevtml-1-1.north east)!0.5!(vtimeline\thevtml-1-2.north west)$)--
            ($(vtimeline\thevtml-\endmtx-1.south east)!0.5!(vtimeline\thevtml-\endmtx-2.south west)$);

        \foreach \x in {1,...,\endmtx}{
            \node[circle,timeline color st, inner sep=0.15pt, draw=white, thick]
            (vtimeline\thevtml-c-\x) at
            ($(vtimeline\thevtml-\x-1.east)!0.5!(vtimeline\thevtml-\x-2.west)$){};
                \draw[timeline color st](vtimeline\thevtml-c-\x.west)--++(-3pt,0);
        }

        \ifvtimelinetitle%
            \draw[timeline color st]([yshift=\lineoffset]vtimeline\thevtml.north west)--
                ([yshift=\lineoffset]vtimeline\thevtml.north east);

            \node[anchor=west,yshift=16pt,font=\large]
                at (vtimeline\thevtml-1-1.north west)
                {\textsc{Timeline \thevtml}: \textit{\vtimelinetitle}};
        \else%
            \relax%
        \fi%

        \ifvtimebottomline%
            \draw[timeline color st]([yshift=-\lineoffset]vtimeline\thevtml.south west)--
            ([yshift=-\lineoffset]vtimeline\thevtml.south east);
        \else%
            \relax%
        \fi%
    \end{tikzpicture}
}

% Sample to use in the document
% 
% \begin{vtimeline}[description={text width=7cm},
%     row sep=4ex,
%     use timeline header,
%     timeline title={The title}]
%     1947 & AT and T Bell Labs develop the idea of cellular phones\endlr
%     1968 & Xerox Palo Alto Research Centre envisage the `Dynabook'\endlr
%     1971 & Busicom 'Handy-LE' Calculator\endlr
%     1973 & First mobile handset invented by Martin Cooper\endlr
%     1978 & Parker Bros. Merlin Computer Toy\endlr
%     1981 & Osborne 1 Portable Computer\endlr
%     1982 & Grid Compass 1100 Clamshell Laptop\endlr
%     1983 & TRS-80 Model 100 Portable PC\endlr
%     1984 & Psion Organiser Handheld Computer\endlr
%     1991 & Psion Series 3 Minicomputer\endlr
% \end{vtimeline}
% or
% \begin{vtimeline}[timeline color=black!80!gray, line offset=2pt]
%     1947 & AT and T Bell Labs develop the idea of cellular phones\endlr
%     1968 & Xerox Palo Alto Research Centre envisage the `Dynabook'\endlr
%     1971 & Busicom 'Handy-LE' Calculator\endlr
%     1973 & First mobile handset invented by Martin Cooper\endlr
%     1978 & Parker Bros. Merlin Computer Toy\endlr
%     1981 & Osborne 1 Portable Computer\endlr
%     1982 & Grid Compass 1100 Clamshell Laptop\endlr
%     1983 & TRS-80 Model 100 Portable PC\endlr
%     1984 & Psion Organiser Handheld Computer\endlr
%     1991 & Psion Series 3 Minicomputer\endlr
% \end{vtimeline}

%%  ========================================================================
%%  Code highlight
%%  ========================================================================

\usepackage{listings}

\newfontfamily{\codefont}{Input Mono}    % font for code snippets

\definecolor{dkgreen}{rgb}{0,0.6,0}
\definecolor{red}{rgb}{0.9,0.1,0.1}
\definecolor{gray}{rgb}{0.5,0.5,0.5}
\definecolor{mauve}{rgb}{0.58,0,0.82}

\lstset{
  frame=single,
  framesep=0.25cm,
  framexleftmargin=0cm,
  framexrightmargin=0cm,
  rulecolor=\color{lightgray},
  backgroundcolor=\color{lightgray!20},
  aboveskip=3mm,
  belowskip=3mm,
  showstringspaces=false,
  columns=flexible,
  basicstyle={\codefont\fontsize{9}{10}\selectfont},
  numbers=none,
  numberstyle=\tiny\color{red},
  keywordstyle=\color{blue},
  commentstyle=\color{dkgreen},
  stringstyle=\color{mauve},
  breaklines=true,
  breakatwhitespace=true,
  postbreak=\mbox{\textcolor{red}{$\hookrightarrow$}\space},
  tabsize=3,
  framextopmargin=0pt,
  framexbottommargin=0pt,
}

%%  ========================================================================
%%  Questions
%%  ========================================================================

\usepackage{enumitem}

\newlist{questions}{enumerate}{3}
\setlist[questions]{itemindent=1cm, align=parleft,itemsep=0.25cm,parsep=\lineskip}
\setlist[questions,1]{label=Q\,\arabic*,leftmargin=1.5\parindent,labelindent=0pt,ref=\arabic*}
\setlist[questions,2]{label=Q\,\arabic{questionsi}.\arabic*,leftmargin=.5\parindent,labelindent=-1.5\parindent,ref=\arabic{questionsi}.\arabic*,topsep=\lineskip,partopsep=\lineskip}
\setlist[questions,3]{label=Q\,\arabic{questionsi}.\arabic{questionsii}.\arabic*,leftmargin=.5\parindent,labelindent=-2\parindent,ref=\arabic{questionsi}.\arabic{questionsii}.\arabic*,topsep=\lineskip,partopsep=\lineskip}
\newcommand*\qn{\stepcounter{cntquestions}\item\label{qn:\thecntquestions}}
\newcommand*\ans{\item[A\,\ref{qn:\thecntquestions}]}
\newcounter{cntquestions}

%%  ========================================================================
%%  Additional settings
%%  ========================================================================

% \usepackage{showframe}                  % shows the document framing
\usepackage{blindtext}                  % generates placeholder text
\usepackage{xcolor}                     % for additional coloring
\usepackage{array}                      % for table styling
\usepackage[hidelinks]{hyperref}        % links

%%  ========================================================================
%%  Document start
%%  ========================================================================

\begin{document}

%%  ========================================================================
%%  Title page
%%  ========================================================================

\begin{titlepage}

    \thispagestyle{fancy} % Use fancy headers with TLP classifications

    \fancyfoot[C]{} % Removes the page numering from the title page

    \vspace*{\fill}

    \begin{flushleft}

        {\Large \textbf{Template}}

        \vspace{0.5cm}
        
        {\large A write-up template by \href{https://github.com/egrzeszczak}{@egrzeszczak}}

        \vspace{0.5cm}

        {\blindtext[1]}

    \end{flushleft}

    \vspace*{\fill}
    
\end{titlepage}

%%  ========================================================================
%%  Content start
%%  ========================================================================

%%      ====================================================================
%%      Table of contents
%%      ====================================================================

\clearpage
\tableofcontents
\clearpage

%%      ====================================================================
%%      Introduction
%%      ====================================================================

\section{Introduction}

{\blindtext[1]}

\clearpage

%%      ====================================================================
%%      Objectives
%%      ====================================================================

\section{Objectives}

{\blindtext[1]}

\vspace{0.5cm}

Questions to answer:

\setcounter{cntquestions}{0}
\begin{questions}
    \qn \label{q:1} Question 1
    \qn \label{q:2} Question 2
    \qn \label{q:3} Question 3
    \qn \label{q:4} Question 4
    \qn \label{q:5} Question 5
    \qn \label{q:6} Question 6
    \qn \label{q:7} Question 7
    \qn \label{q:n} Question N
\end{questions}

\clearpage

%%      ====================================================================
%%      Preparation
%%      ====================================================================

\section{Preparation}

\subsection{Tools}

{\blindtext[1]}

\subsection{Evidence}

{\blindtext[1]}

\vspace{0.5cm}

\begin{center}
{\renewcommand{\arraystretch}{1.7}
\begin{tabular}{| c | c |}
    \hline
    SHA256 & File \\
    \hline
    \small{ffffffffffffffffffffffffffffffffffffffffffffffffffffffffffffffff} & file.img \\
    \hline
\end{tabular}
}
\end{center}

\clearpage

%%      ====================================================================
%%      Analysis
%%      ====================================================================

\section{Analysis}

\subsection{Analysis A}

{\blindtext[3]}

\subsubsection{Subanalysis AA}

{\blindtext[1]}

\subsubsection{Subanalysis AB}

{\blindtext[1]}

\subsection{Analysis B}

{\blindtext[3]}

\subsection{Analysis C}

{\blindtext[1]}

\subsubsection{Subanalysis CA}

{\blindtext[1]}

\subsubsection{Subanalysis CB}

{\blindtext[1]}

\subsubsection{Subanalysis CC}

{\blindtext[1]}

\subsection{Analysis D}

{\blindtext[3]}

%%      ====================================================================
%%      Post-analysis
%%      ====================================================================

\clearpage
\section{Post-analysis}

%%      ====================================================================
%%      Summary
%%      ====================================================================

\subsection{Summary}

{\blindtext[1]}

%%      ====================================================================
%%      Answers
%%      ====================================================================

\subsection{Answers}

\setcounter{cntquestions}{0}
\setcounter{cntquestions}{0}
\begin{questions}
    \qn \label{q:1} Question 1
    \ans \label{a:1} Answer 1
    \qn \label{q:2} Question 2
    \ans \label{a:2} Answer 2
    \qn \label{q:3} Question 3
    \ans \label{a:3} Answer 3
    \qn \label{q:4} Question 4
    \ans \label{a:4} Answer 4
    \qn \label{q:5} Question 5
    \ans \label{a:5} Answer 5
    \qn \label{q:6} Question 6
    \ans \label{a:6} Answer 6
    \qn \label{q:7} Question 7
    \ans \label{a:7} Answer 7
    \qn \label{q:n} Question N
    \ans \label{a:n} Answer N
\end{questions}

\clearpage

%%      ====================================================================
%%      Timeline
%%      ====================================================================

\subsection{Timeline of events}

Time for the event is provided in UTC.

\vspace{0.5cm}

\begin{vtimeline}[timeline color=black!80!gray, row sep=4ex, description={text width=14cm}]
    1947 & AT and T Bell Labs develop the idea of cellular phones\endlr
    1968 & Xerox Palo Alto Research Centre envisage the `Dynabook'\endlr
    1971 & Busicom 'Handy-LE' Calculator\endlr
    1973 & First mobile handset invented by Martin Cooper\endlr
    1978 & Parker Bros. Merlin Computer Toy\endlr
    1981 & Osborne 1 Portable Computer\endlr
    1982 & Grid Compass 1100 Clamshell Laptop\endlr
    1983 & TRS-80 Model 100 Portable PC\endlr
    1984 & Psion Organiser Handheld Computer\endlr
    1991 & Psion Series 3 Minicomputer\endlr
\end{vtimeline}

%%      ====================================================================
%%      Indicators of Compromise
%%      ====================================================================

\subsection{Indicators of Compromise}

\subsubsection{Files}

\begin{center}
{\renewcommand{\arraystretch}{1.7}
\begin{tabular}{| c | c |}
    \hline
    SHA256 & File name \\
    \hline
    \small{ffffffffffffffffffffffffffffffffffffffffffffffffffffffffffffffff} & file.img \\
    \hline
\end{tabular}
}
\end{center}

\subsubsection{IPv4s}

\begin{center}
{\renewcommand{\arraystretch}{1.7}
\begin{tabular}{| c | c |}
    \hline
    IPv4 & Name \\
    \hline
    \small{8.8.8.8} & Google DNS \\
    \hline
\end{tabular}
}
\end{center}

\subsubsection{Domains}

\begin{center}
{\renewcommand{\arraystretch}{1.7}
\begin{tabular}{| c | c |}
    \hline
    Domain & Description \\
    \hline
    \small{google.com} & Google \\
    \hline
\end{tabular}
}
\end{center}

\subsubsection{Certficates}

\begin{center}
{\renewcommand{\arraystretch}{1.7}
\begin{tabular}{| c | c |}
    \hline
    SHA1 Fingerprint & Name \\
    \hline
    \small{aaaaaaaaaaaaaaaaaaaaaaaaaaaaaaaaaaaaaaaa} & Certificate \\
    \hline
\end{tabular}
}
\end{center}

%%      ====================================================================
%%      Aditional resources
%%      ====================================================================

\clearpage

\section{Additional resources}

\blindtext[1]

\end{document}

%%  ========================================================================
%%  Content and document end
%%  ========================================================================
